\newpage
\chapter{Стан наукової проблеми}


Organic–inorganic sol–gel hybrid compounds combining useful properties of both organic and inorganic components have attracted much attention during the last decades. Much progress has been attained in creation of new multifunctional materials and comprehension of the underlying chemistry and physics. The seemingly unlimited variety, unique structure–property control, and the compositional and shaping flexibility give these hybrid materials a high potential. Indeed, functional hybrid materials have had an explosive development since the eighties and some of them are already commercial. Many studies have shown that optical properties of the hybrid materials can be considerably enhanced as compared to organic polymer materials, making them promising systems for nanophotonic and nanobiophotonic applications in the coming years.

Although most research is done on silica-based hybrids, titania-based hybrids deserve an increased interest. $TiO_2$ is usually added to hybrid composition to improve mechanical properties and optical properties: (i) linear—addition of TiO2 changes the refractive index and (ii) nonlinear-addition of $TiO_2$ increases NLO $\chi^{(3)}$ properties of the material. UV radiation can initiate polymerization of unsaturated organic bonds allowing densification and shaping of the hybrid network. Moreover, the photo-induced polymerization of the hybrid solutions is used for fabrication of surface-relief gratings, details of micrometre sizes, and 2D and 3D photonic structures with micrometric periodicity.

Less attention has been paid to the photochromic properties of organic–inorganic $TiO_2$-based hybrids. We have recently demonstrated that the inorganic component in $TiO_2$ wet-gels offers a high efficiency towards photo-induced charge separation and long-term storage of electrons in form of small polarons. The reported $TiO_2$ wet-gels were prepared by the polycondensation in acidic conditions of titanium alkoxides in alcohol. The electron transport process can be initiated by laser photons, in particular, and the high photo- activity of these ‘‘alcogels’’ can be explained by a macroscopic inorganic bulk structure, which differs from that of conventional solids by the existence of an extended liquid/solid hybrid interface. Despite the high photosensitivity of the wet $TiO_2$ alcogels, their poor mechanical toughness restricts their potential applications. Consequently, high photonic sensitivity inherent to the inorganic component has been lately reported on $TiO_2$ hybrid materials including two macroscopic interpenetrating (i) inorganic oxo-titanate, and (ii) organic poly(hydroxyethyl methacrylate) PHEMA, networks. These nanocomposites are obtained without shrinkage, have high transparency, and possess sufficient thermal stability and mechanical toughness to allow optical-grade surface polishing. The extended coupling between the inorganic and organic backbones assures rapid holes scavenging and long-term trapped electron stability and their ability to 3D-laser microstructuring has been demonstrated.

In wet $TIO_2$ alcogels, the liquid phase (alcohol) enables efficient hole scavenging, in contrast to conventional $TIO_2$ solids where the hole remaining in the material bulk provokes a rapid charge annihilation process. Moreover, external agents like oxygen heal surface $Ti^{3+}$ sites. The organic–inorganic hybrids may take benefit of a large interface between the two components, which is able to strongly affect their electronic structure and electron transport. The existence of efficient and processing-tunable interfacial electron transport in $TIO_2$ hybrid materials constitutes the main issue of the present study. Up to now no comparative studies of $TIO_2$ hybrids have been performed in order to inspect the influence of their nano- and microstructure on photonic sensitivity. This kind of study represents an indispensable work for the design of tailor- made sol–gel-derived hybrid materials. The present work is a first step in this direction.
