
\newpage
\chapter*{Вступ}
\addcontentsline{toc}{chapter}{Вступ}

Впродовж останнього десятиліття було досягнуто великого прогресу як в синтезі нових функціональних органічно-неорганічних гібридних матеріалів так і в розумінні їх фізичних властивостей. Велика різноманітність таких гібридів розвивалися для застосування в фотоніці, датчиках, біомедицині, фото-каталізі, сонячних елементах і т.п. Серед них, гібриди на основі діоксиду титану викликають особливий інтерес головним чином через біологічну сумісність, електропровідні, механічні та оптичні властивості.

Матеріали на основі діоксиду титану можуть бути використані в якості пігментів, порошків для каталітичних або фотокаталітичний додатків, як колоїди і тонкі плівки для фотоелектричних, електрохромних, фотохромних, електролюмінесцентних приладив і датчиків, в якості компонентів для просвітлюючих покриттів, як пористі мембрани для ультрафільтрації або навіть як вогнетривке волокно та ін.

%В даній роботі досліджено зміну нелінійно-оптичних властивостей органо-неорганічних властивостей гібридних матеріалів на основі pHEMA
\clearpage
